%
% RING Meeting Template
% Edit and compile with pdflatex
%
% If you absolutely want to use latex to produce DVI, be sure to use only
% compatible graphisms (.eps is the best).
% You can also include .eps figs only and use pdflatex (thanks to epstopdf),
% just check that you have "shell_escape = 1" somewhere in a texmf.cnf file,
% or that you call "pdflatex -shell-escape file.tex".
%
% This template uses natbib.sty bibliography style, so you can use commands
% like \citet{}, \citep{}, \citeauthor{} or \citeyear{}.

%-----------------------------------------
% Template Mode preprint / final:

\documentclass[preprint]{ring20} 

% Please : submit your paper in the final mode
%\documentclass[final]{ring20} 

%% relative path to the graphic folder from the tex file
\graphicspath{%
{./Figures/}%
}

\usepackage[disable]{todonotes}
%\usepackage[]{todonotes}

\newcommand{\bx}{\mathbf{x}}
\newcommand{\bn}{\mathbf{n}}

%-----------------------------------------
% First page information :

\title{Progresses in the implicit structural modeling benchmark}

% Title used for the title over-riding report
% (It may be the same than the main title if small enough)
\shorttitle{Implicit structural modeling benchmark}

% Author(s) name(s) on the first page
\author[1]{Guillaume Caumon}
\author[1,2]{Julien Renaudeau}
\author[1]{Modeste Irakarama}
\author[3]{Lachlan Grose}
\author[5]{Miguel de la Varga}
\author[6]{Michael Hillier}
\author[6]{Eric de Kemp}
\author[5]{Florian Wellmann}
\author[1]{Pauline Collon}
\author[4]{Gautier Laurent}
\author[3]{Laurent Ailleres}
%\author[8]{Who else ? Order to be determined}

%Adresses of the authors (affiliations)  
\affil[1]{RING, GeoRessources / ENSG, Universit\'e de Lorraine / CNRS, France}
\affil[2]{Schlumberger, France}
\affil[3]{Monash University, Australia}
\affil[4]{Univ. Orl\'eans, CNRS, BRGM, ISTO, France}
\affil[5]{RWTH Aachen, Germany}
\affil[6]{NRCan, Canada}

%often used affiliations
%\affil[1]{GeoRessources - UL/CNRS/CREGU, ENSG, Vandoeuvre-l\`es-Nancy, France.}
%\affil[2]{GeoRessources - UL/CNRS/CREGU, ENSMN, Nancy, France.}
%\affil[3]{Centre d'hydrog\'eologie et de g\'eothermie, Universit\'e de Neuchatel, Neuchatel, Suisse.}
%\affil[5]{INRIA - Project Alice, Villers-l\`es-Nancy, 54600, France.}

% Author(s) name(s) used in the footer of each page
\shortauthor{Caumon et al.}

%-----------------------------------------
% The document :
%

\begin{document}
\maketitle

\begin{abstract}

Implicit methods have gained significant popularity in recent years as 
an alternative to surface-based structural modeling methods (also known as 
contouring, gridding or explicit structural modeling). These approaches represent 
geological interfaces as equipotentials of a scalar field. 
Whereas existing formulations share the same principle, they have not yet been 
quantitatively compared beyond theoretical discussions. We propose a new review 
 of existing methods and highlight their main features in terms 
of data handling, basis function, mathematical principles and relationships to 
meshing. We propose three benchmark data sets which present specific 
challenges for implicit modeling: a folded turbidite (Hecho), a carbonate buildup 
displaying large thickness variations (Claudius) and a convoluted synthetic surface 
representing a hydrothermal body (Moureze). We applied several implementations of the 
various classes of implicit methods on these data sets, using one single scalar field and minimal parameter 
tuning. Results are available on an open repository. We propose metrics to analyze 
the obtained results in terms of data accuracy, ability to predict 
some features not present in the data, and topological complexity. 
We highlight, however, the risk of abusing these metrics to provide a definitive 
global ranking of methods or codes. 
The first results highlight show, for all methods, inconsistent stratigraphic features 
on the Claudius data. Even though this can be addressed using several scalar fields, 
this illustrates the difficulty to come up with a universal geological 
modeling method, and highlights the need to build additional geological 
knowledge to cope with uncertainty in future structural modeling methods.  

\end{abstract}

\todo[inline]{Writing mode: I suggest a ``dictatorial collaborative writing'': co-authors send me comments / paragraphs / figures / results in annotated PDFs, LateX (or Word) format and I will incorporate them in the LaTeX document.

Authors: list and order to be discussed. }


\todo[inline]{Eric: "I was thinking of developing  further a 3d matrix with principal axis indicating model  level of complexity (dimension+properties, geological events, cumulative spatial features) as well as a matrix indicating data complexity (density, number of types, clustering)  . Each test case could fit into these two matrices and could be increased in dimensionality but is at least more intuitive where the case examples fall visually. Each case would get a complexity badge or glyph classifying it."}

\textbf{Preamble: } \textit{this paper an update from the 2019 RING Meeting version. As compared to last year' version, the introduction has been edited to better frame the objectives, the theoretical review of the methods has been enriched, and the evaluation criteria have been added. Please note that this version has not yet been reviewed by all co-authors. Updates will be available at \url{https://github.com/Loop3D/ImplicitBenchmark/tree/master/Paper}.}

\section*{Introduction}

The formation and deformation of rocks at geological time scales generate a large variety of geological structure shapes. Three-dimensional structural modeling aims at reconstructing these structures from spatial samples created from field observations and/or from geophysical images of the subsurface. Historically, structural modeling first meant computing the depth contours of a particular geological interface \citep[e.g.,][]{Walters1969AB,Hardy1971JGR,Briggs1974G,Bolondi1976G}. During the 1990's surface-based models using polygonal or parametric surfaces emerged as a more powerful way to represent complex structures such as recumbent folds or salt diapirs \citep{Mallet1992CD,deKemp1999CG}. However, these explicit modeling approaches are difficult to automate in complex geological settings, and call for various degrees of interactive expert input to generate the structures. Since then, progress in memory and computational capabilities have led to implicit modeling formulations, in which geological interfaces between rock units are defined by isovalues of a three-dimensional scalar field \citep[][]{Lajaunie1997MG,Cowan2002ASGMEM,Calcagno2008PEPI,Frank2007CG,Caumon2013GaRSITo,Souche20137ECEISE2,Hillier2014MG,Laurent2016MG,Laurent2016EaPSL,Martin2017CG,Grose2017JSG,delaVarga2018GMDD,Irakarama2018EAGE,Grose2019JoSG,Renaudeau2019BEMRMX,Renaudeau2019MG,Manchuk2019CG}. This diversity of methods is \textit{per se} a motivation for the present contribution. 

Implicit modeling provides a high degree of automation even for complex geological shapes, and can also account for various types of structural data such as off-contact orientation  measurements. As a result of these methodological advances, three-dimensional structural modeling is increasingly becoming part of the geologist's toolbox. By integrating sparse data into a consistent graphical model, the promise of computational structural modeling is to help geologists to analyze their data, to develop conceptual interpretive models, and to make subsurface forecasts based on model outcomes. 

Automation of three-dimensional structural modeling is essential to meet these needs effectively, especially when several possible models need to be considered to appropriately characterize subsurface uncertainty \citep{Wellmann2018AiG}. However, ``seeing is believing'', so a risk exists that applied geologists believe that a particular structural model is true, whereas it is only determined from the available spatial data (whose reliability may vary) and from mathematical or numerical terms telling what the three-dimensional model should look like away from the data points. Indeed, in general, the data points are not numerous enough to fully characterize the structural geometry, hence the need for some sort of regularization \citep[e.g.,][]{Renaudeau2019MG}. The parameters involved in the regularization term influence the results, but may not be easy to determine by layman users. These parameters oftentimes correspond to abstract mathematical terms (e.g., least-squares weights, variogram ranges, basis function parameters) which can be intuitively translated into model features (e.g., degree of data fit, stiffness of the surface) but not so easily into classical structural concepts, which generally involve a blend of geometric, kinematic, and mechanical reasoning. Analyzing and understanding the results of a particular method can be difficult, as it may call for the practitioner to delve into the theory of partial differential equations, numerical optimization, or geostatistics. Another approach to deal with data ambiguity and uncertainty is to actually consider multiple structural models. However, all structural uncertainty modeling approaches rely on a deterministic structural modeling engine \citep[see ][ and references therein]{Wellmann2018AiG}.

To address these challenges, several authors have proposed to more directly integrate structural concepts into modeling methods \citep{DeKemp2003G,Maxelon2009CG,MassiotGM2010,Laurent2016EaPSL,Grose2017JSG,Grose2018JGRSE,Grose2019JoSG}. This can be achieved either by manually or automatically adding new interpretive ``data'' such as fold hinges, or by integrating new mathematical terms in the interpolation problem. Another recent line of research has been on the mathematical formulation and the discretization of the regularization term \citep{Laurent2016MG,Martin2017CG,Irakarama2018EAGE,Renaudeau2019MG}. In this frame, theoretical comparisons between the various published methods can be made, but they may be of limited use to practitioners. Practical comparisons between method are delicate, because they rely on software packages whose implementation is not necessarily available. The level of implementation descriptions may also vary from one paper to another, making it difficult for graduate students to reproduce previous work. 

In this paper (Section \ref{sec:methods}), we propose a synthetic review of the various interpolation methods which form the basis of implicit structural modeling software. The goal of this presentation is to use consistent notations for all methods, with the intent to more easily highlight similarities and differences between these various methods. We briefly summarize the features and variants of six computer implementations of these methods, and stress the main parameters which can be used to adapt them to a particular geological setting. As in machine learning, tuning these method parameters using validation data can be achieved in principle. However, the general lack of data may entail a poor predictive ability for parameter-rich models (overfitting); conversely, parsimonious models may be more stable but have a lower fit. This may explain why, in practice, parameter tuning is seldom done in three dimensional structural modeling. Another reason is that defining a goodness criterion is not immediate in the case of geological models. Geologists are trained to decide on the quality of a particular model based on their expertise, and are often reluctant to use purely data-driven methods for that. Expressing in mathematical terms the ``geological realism'' of a particular structural model is possible under some conditions, but it can be difficult to automate and, in general, is still an open problem \citep{Caumon2010MG,Jessell2010T}. 

The above remarks motivate the realization of the present benchmark study, which is the first community effort of its kind in the field of three dimensional structural modeling. To keep this first study tractable, we limit ourselves to three benchmark data sets in various geological contexts. Similar benchmark data exist in the computer graphics community \citep[e.g.,][]{BLNTS13}, but they generally aim at reconstructing only one single manifold surface from a relatively dense point cloud. In geosciences, the difficulty to access the subsurface generally entails more sparsity and more diversity in the field data. In particular, stratigraphic orientations are often available from fieldwork and dipmeter measurements in boreholes. Another difficulty specific to geological modeling is the presence of faults, and the simultaneous reconstruction of several conformable stratigraphic surfaces with just one scalar field, which raises questions about the management of thickness variation \citep{Laurent2016MG}. The data sets presented in this paper (Section \ref{sec:data}) account for these geological peculiarities and propose several test cases of variable complexity. Each case includes basic data types (points and orientations). To keep the comparison simple, more advanced data types such as fold axis, foliation data, and structural lineations are not considered in the present paper. Some faults are present in two data sets because they were observed, but they are local and only have a limited displacement. Considering how the various interpolation methods manage faults is outside the scope of this paper. 

In Section \ref{sec:results}, we propose a set of metrics to compare results, and explain why and how these metrics were selected. The goal of this benchmark is not to provide a general ranking of the various interpolation methods. Rather, the goal is to gain insights about the various methods on a set of specific problems. Indeed, measuring some aspects of the results quality is always useful and may help defining avenues for future research. Whereas a global ranking of methods based on these metrics could be tempting, we do not recommend it, because metrics are incomplete and results obtained are case-specific and non-exhaustive. Also, we deliberately chose in this paper to reflect the current practice of geological modeling by a applying a set of implicit modeling methods without parameter or hyperparameter tuning. As a results, the full potential of the various methods is certainly not reached. Finally, the data points come from the interpretation made by one of the co-authors, hence they are also affected by uncertainty and may also be biased; evaluating their accuracy is not straightforward. For these reasons, the generalization of the method's performance to other geological domains or data configurations should not be taken for granted. Nonetheless, we expect that the application and proposed metrics to bring insights about the methods and their suitability to provide a reasonable approximation of three geological domains from incomplete observations. 

For future extensions and completeness, we have placed the data and results on a public repository at \url{https://github.com/Loop3D/ImplicitBenchmark/}. This makes it possible to scrutinize the results, to add new assessment metrics and also to add new modeling results and new data sets in the future. SKUA Macros implementing the proposed analysis metrics and some Jupyter notebooks are also available on the repository to help with the reproducibility of the metrics computation. 

%Software availablility and usability are important, sability issues, even though they can important, but we will focus on  Among these, the ability of a particular interpolation method to interpolate or extrapolate correct structural geometries away from data points can always be challenged. The adequacy of automatic methods has been debated since the early times of automatic map contouring \citep{Walters1969AB}. The data density should be such that results are well constrained.  ideally, a good method should be able produce results similar to those of a skilled structural geologist.  but this can be difficult to quantify.  . In principle, the goodness of an interpolated structural geometry can be measured on a particular data set by cross-validation (ie: computing the error between the forecasted geometry and some randomly chosen data points). However, as in machine learning, the tuning parameters of a particular method can be optimized using cross-validation, raising the risk of overfitting. Philosophically, 

\todo[inline]{
Previous comments: 

PC: "this supposes there is a "reference true" surface? I do not like it very much... 
Shoudn't we  keep the analysis focused on what we know (data, geological concepts ?). 
In this case, I would prefer "ability to consistently extrapolate the surface"... 

MI: "- What is "quality" in this context and how do we measure it?
- About the ability to predict the reference surface away from data, as we discussed some time ago, I think this is questionable.
  In practice, I can think of two cases: (1) Either we have an a priori on that reference surface away from the data  and therefore we should provide that information to our interpolator.
  Otherwise we know in advance that the interpolator will most likely not predict the surface away from the data because we explicitly ask the interpolator for a "smooth" surface.
  (2) Either we do not have an a priori on that reference surface and therefore we just have to accept what the interpolator gives us. In this second case, different interpolators
  might give different results. We can certainly compare and comment on these results but I don't see how we can decide which are "better" since we are in a case of no a priori. " 
}

\section{A review of implicit modeling methods}
\label{sec:methods}

\todo[inline]{GC: I have expanded this section as compared to last year, but not up to a very detailed description of every method. Instead, I have tried to summarize the main ideas and give the key equations. This section may still contains some approximations and errors, and probably need editing to improve clarity by a general readership. Please have a read and comment and don't hesitate to make (constructive) suggestions. } 

We now propose a brief review of the main methods proposed to compute three-dimensional scalar fields in implicit geological structural modeling. 
Consider $S$ as the set of three-dimensional positions $\bx$ where a three-dimensional scalar field $f(\bx)$ takes the value $f_S$: 

\begin{equation}
\label{eq:levelset}
  S = \{\mathbf{x} | f(\mathbf{x}) = f_S\}.
\end{equation}

In the general case of a spatially variable scalar field $f$, these positions form a two-dimensional manifold surface called iso-surface or level set, which represents geological surfaces in implicit methods. 

The expression (\ref{eq:levelset}) can be extended to represent 
at once a stratigraphic series of $K$ conformable horizons 
$S_1, \ldots, S_K$ using different iso-values $f_1, \ldots, f_K$. 
In the following, we will call $f$ the scalar field or the stratigraphic function, as $f$ can be seen 
(in a chronostratigraphic sense) as a relative geological time \citep{Mallet2004MG,Lomask2006G,Wu2012G}.

In the presence of $N_k$ data points at positions $\{\mathbf{x}_1^k, \ldots, \mathbf{x}_{N_k}^k\}$ for the horizon $S_k$, all implicit surface interpolation methods start by forming the following system of equations: 

\begin{equation}
\label{eq:datasystem}
\begin{array}{r c l}
f(\bx_1^1) & = & f_1 \\
\vdots & \vdots & \vdots \\
f(\bx_{N_1}^1) & = & f_1 \\
\vdots & \vdots & \vdots \\
\vdots & \vdots & \vdots \\
f(\bx_{N_1}^K) & = & f_K \\
\vdots & \vdots & \vdots \\
f(\bx_{N_K}^K) & = & f_K \\
\end{array}
\end{equation}

The system (\ref{eq:datasystem}) can be complemented with other data terms corresponding to $N_O$ orientation measurements. These orientations, made at locations 
$\bx_o$, $o = \{1, \ldots, N_O\}$, may take the form of unit strike and dip vectors $(\mathbf{s}_o, \mathbf{d}_o)$ or unit normal vectors $\bn_o$, with $\mathbf{n}_o = \mathbf{d}_o \times \mathbf{s}_o$.  
 
From this information, several methods have been proposed to compute a stratigraphic function at any spatial location $\mathbf{x}$. Essentially, these methods make different choices about the way data are handled, the type of basis function used to approximate the solution, the mathematical formulation to ensure the uniqueness of the solution. These differences have an impact on the way the solution is evaluated and visualized. 
 
\subsection {Data handling}

In most approaches, the scalar field values $f_k$ need to be chosen before the interpolation: they are imposed in the interpolation problem \citep[e.g., ][]{Frank2007CG,Hillier2014MG,Renaudeau2019MG}. In practice, these values can be chosen using a rough estimate of the cumulative thickness between each horizon and some reference horizon, or any other convenient value proportional to that cumulative thickness \citep{Caumon2013GaRSITo,Collon2016I}. 

Orientation information can be integrated either by equating the normal to the scalar field gradient ($\nabla f_{\bx_o} = \mathbf{n}_o$, which  carries information about the stratigraphic polarity and assumes that the interpolated scalar field approximates the signed distance to some reference horizon), or by equating the dot products between the scalar field gradient and strike and dip gradients ($\nabla f_{\bx_o} \cdot \mathbf{s}_o = \nabla f_{\bx_o} \cdot \mathbf{d}_o = 0$, which does not use polarity and does not carry any assumption on the norm of the scalar field gradient). 

To avoid the pre-estimation of the horizon values $f_k$, dual cokriging \citep[DcK, ][]{Lajaunie1997MG,Chiles04OMSMP} performs a substitution in System (\ref{eq:datasystem}) so as to consider (unknown) increments of the scalar field as compared to one reference horizon instead of the absolute isovalues. The primal form of the cokriging system reads

\begin{equation}
\label{eq:DcK}
f(\bx) - f_{ref} = \sum_{o=1}^{N_O} (\lambda_i \bn_o^x + \mu_i \bn_o^y + \nu_i \bn_o^z) + \sum_{k=1}^{K} \sum_{j=2}^{N_K} \lambda_{j}^{k}[f(\bx_j^k) - f(\bx_1^k)], 
\end{equation}

where $f_{ref}$ is an arbitrary value chosen for one of the horizons, $k$ refers to the horizon index, $j$ is the point index on each horizon (up to a permutation), and $\lambda_i$, $\mu_i$, $\nu_i$ and $\lambda_j^k$ are the unknown kriging coefficients. In Eq. (\ref{eq:DcK}), the orientation data $\bn_o$ are treated as secondary data in the estimation of the scalar field. The cross-covariance between the scalar field and its gradient and the covariance of the gradient, are determined analytically from the chosen covariance function. For the problem (\ref{eq:DcK}) to be well-posed, at least one orientation data point must be available. 

Other possible data that have been integrated in the interpolation system relate for instance to tangent data \citep{Lajaunie1997MG, Caumon2013GaRSITo, Hillier2014MG}, fold axis \citep{MassiotGM2010,Hillier2014MG}, fold vergence \citep{Laurent2016EaPSL,Grose2017JSG} and layer thickness \citep{Laurent2016MG}. For the sake of brevity and for the homogeneity of the application, we do not cover these additional data types in this paper. 

\subsection{Basis functions} 

The data terms described above only provide information at data location. To evaluate, process or visualize the scalar field everywhere on a computer, numerical methods rely on the general formulation 

\begin{equation}
\label{eq:basis}
  f(\mathbf{x}) = \sum_{i=1}^{N}{v_i\varphi_i(\mathbf{x})},
\end{equation}
\noindent where $v_i$ are some coefficients and $\varphi_i(\mathbf{x})$ some basis functions \citep[e.g., ][]{Hillier2014MG,Renaudeau2019MG}.

Note that this representation is so general that it also applies to the description of parameter fields used in thermo-hydro-mechanical and chemical process models. 
In that sense, creating a subsurface model essentially amounts to choosing the basis functions deemed appropriate to parameterize the subsurface, and to find the unknown coefficients based on the available data \citep{Caumon2018HoMG,Wellmann2018AiG}.
 
In radial basis function interpolation \citep[RBF, ][]{Carr2001,Cowan2002ASGMEM,Hillier2014MG} and dual co-kriging \citep{Lajaunie1997MG}, the basis functions (or covariance functions) are centered on the data points, and a global polynomial term is often considered to capture the overall trend over the domain. In this case, $N$ is equal to the total number of data points plus the number of terms $N_P$ of the polynomial ($N = N_P + N_O + \sum_{k=1}^{K}{N_k}$). In both cases, the choice of the basis function will govern the shape of the iso-surfaces of the interpolated scalar field. As opposed to classical applications of Kriging, inferring the type and parameters of the covariance function from the data is difficult by the non-stationarity and the increment formulation. Therefore, the choice of the covariance function is mainly heuristic, with the mathematical requirement that the covariance function must be twice differentiable to compute the covariances $Cov\left(f(\bx), \nabla {f}(\bx+\mathbf{h})\right)$ and $Cov\left(\nabla f(\bx), \nabla f(\bx+\mathbf{h})\right)$. 

In this study, we use the code GemPy \cite{delaVarga2018GMDD} which implements DcK, and uses the cubic covariance function for its  robustness and its ability to provide coherent results in many studies: 

\begin{equation}
C(r) = \begin{cases}
C_0(1-7(\frac{r}{a})^2+ \frac{35}{4}(\frac{r}{a})^3
- \frac{7}{2}(\frac{r}{a})^5 +\frac{3}{4}(\frac{r}{a})^7) &
\text{for } 0  \leq r \leq a \\
0 & \text{for } r  \geq a
\end{cases},
\end{equation}

\noindent where $r$ represents the distance between the points (e.g., between the position $\bx$ to estimate and the data point $\i$ in Eq. (\ref{eq:basis})), $a$ is the range (the maximum influence distance of a data point) 
and $C_0$ the variance of the data. In GemPy, it is recommended to choose a range value close to the
maximum model extent to avoid artifacts in the solution. There is no strong guideline 
to choose the values of the covariance at 0, as changing this value only weights the relative influence of the values and of the gradient data, hence does not affect the shape of the isosurfaces. As in classical kriging, it is also possible to add a nugget effect to the covariance function to transform the exact interpolation into an approximation. In this case, the nugget value relatively to $C_0$ controls the degree of approximation (exact if zero, polynomial if equal to $C_0$. This feature is used in two other codes used in this study: SurfE \citep{Hillier2014MG} and MSS {\citep{Renaudeau2019MG}. 

The SurfE code \citep{Hillier2014MG}, used in this study also implements the increments interpolation approach. It allows for testing many different types of basis function. When applying SurfE to the benchmark data sets, we tested the cubic basis function 
\begin{equation}
\varphi_{C}(r) = r^3
\end{equation}
and a multiquadric basis function
\begin{equation}
\varphi_{MQ}(r) = (\varepsilon^2 + r^2)^{3/2},
\end{equation}

where $\varepsilon$ controls the flatness of the basis function as the distance $r$ increases. \todo[inline]{GC: Needs clarification.}

In discrete smooth interpolation \citep[DSI, ][]{Frank2007CG,Caumon2013GaRSITo,Souche20137ECEISE2,Laurent2016MG,Irakarama2018EAGE}, the number of basis functions $N$ refers to the number of nodes of the mesh used to approximate the solution. In tetrahedral formulations, each basis function $\varphi_i$ is a piecewise linear ``hat function'' (equal to 1 at node $i$, linearly decreasing on the tetrahedra surrounding node $i$ and equal to zero at all the other nodes). These basis functions imply a piecewise constant gradient of the scalar field. They are used in two codes: StructuralLab \citep{Frank2007CG,Caumon2013GaRSITo} and LoopStructural. 
In the SIGMA code \cite{Irakarama2018EAGE}, Lagrange polynomials defined on a Cartesian grid are used to discretize the data. The basis functions which control the smoothing term are estimated using finite differences. LoopStructural and MSS also implement some finite difference operators, for which the type of basis function is not explicitly defined. 

As recently discussed by \citet{Renaudeau2019MG}, the choice of the basis function used have a strong impact on the structure of the linear system that needs to be solved to evaluate the coefficients $v_i$ in Eq. (\ref{eq:basis}): a relatively small but dense system is formed in the case of RBF and DcK, whereas a larger but sparse system is formed in the case of piecewise linear DSI. Recently, \citet{Renaudeau2019MG} and \citet{Manchuk2019CG} proposed to use moving least squares (MLS) polynomial basis functions centered on some interpolation points spread over the domain of interest. As MLS functions have a local support, data points only influence a finite number of interpolation points, and the solution does not explicitly depend on a mesh. This is the main approach used in the MSS code. 

\subsection{Mathematical principles} 

Several mathematical frameworks have been proposed to derive the interpolation systems and to ensure the uniqueness of the solution. 
All methods rely to some extent on (constrained) optimization and generalized regression to interpolate or approximate the scalar field. Classically, kriging estimates the unknown value as a linear combination of surrounding known values, see Eq. (\ref{eq:DcK}). The coefficients then depend on the chosen variogram model and are obtained by minimizing the estimation variance under the unbiasedness condition. Dual kriging is based on a rewriting of the kriging estimator in the same form as (\ref{eq:basis}), which calls for explicitly inverting the covariance matrix \citep{Lajaunie1997MG,Chiles04OMSMP}. The form of that system is similar to that of RBF interpolation \citep{Dubrule1984CG,Hillier2014MG}. The latter is obtained by applying (\ref{eq:basis}) to all data points, under the constraint that the solution should be orthogonal to the polynomial term \citep{Carr2001,Hillier2014MG}. In the DcK and RBF systems, the number of unknowns is such that the problem is well-posed, possibly by adding Lagrange multipliers. It should be noted that the RBF solution can, in some cases, be efficiently approximated on a discrete Cartesian mesh by a simple iterative algorithm \citep{Park2006ITVCG}. 

In DSI, the solution is obtained by a weighted least-squares formulation where the data term (\ref{eq:datasystem}) is applied on mesh-based basis functions (\ref{eq:basis}). As the number of mesh nodes is generally larger than the number of data, a regularization term is added to the system, based on a minimization of the local variation of second derivative of the scalar field $f$. Several quantities representing this second derivative have been proposed, for instance: the minimization of the gradient difference between two adjacent tetrahedra \citep{Frank2007CG}; discrete Laplacian \citep{Irakarama2018EAGE}; sum of one-dimensional second derivatives approximated along several directions using finite differences \citep{Irakarama2018EAGE}; thin plate bending energy discretized with MLS polynomial basis functions \citep{Renaudeau2019MG}. Note that, although DSI in implicit modeling is often assimilated to the constant gradient regulatization in piecewise linear tetrahedra, its formulation allows for many other regularization possibilities \citep{Mallet1992CD}. Indeed, the original DSI formulation starts from a graph-based representation of the mesh nodes and considers a general class of roughness terms based on weighted neighborhood values to ensure convergence. Alternatively, it is possible to use the same approaches as for solving partial differential equations, for instance the finite element method (FEM). Data then play the role of boundary conditions and the continuous energy formulation to be minimized plays the role of the regularization in DSI \citep{Renaudeau2019MG}. In both cases, an important parameter is the weight penalizing the regularization term against the data term in the least-squares system. 

To summarize, all methods are looking for a ``simple'' solution while honoring the information content provided by the input points. Although the criteria of simplicity vary, they all amount to obtaining a regular solution in terms of second order derivatives. In that sense, a layer cake with flat and isopach layers is certainly optimal for all formulations. \todo[inline]{GC: Shall I add a simple layer cake for everyone to try to make sure that the above sentence is correct?}

Although they may belong to different conceptual frameworks, two modeling methods can be shown to be equivalent in some specific cases. For instance, choosing a linear RBF is equivalent to minimizing the isotropic biharmonic thin spline energy \citep{Carr2001}. In our understanding, this relates to some basis functions corresponding to the fundamental solution of a partial differential equation. Similarities between some DSI formulations and continuous energy formulations have also been highlighted \citep{Renaudeau2019MG}. However, comparisons are made difficult by the diversity of variants which have been proposed within each class of method. The correspondence between the various method parameters is, therefore, difficult to establish. Implicit or explicit criteria about the mathematical quality and rigor of a particular approach may be defined, but they do not necessarily correspond to the experience of practitioners, which also depends on experience, user interface and the ability of a particular implementation to effortlessly produce a model deemed geologically realistic. The above method are clearly data-driven, and none pretends to explicitly represent the succession of geological processes which actually generated the structures \citep{Caumon2010MG,Jessell2010T}. Nonetheless, there is room to make progresses in understanding what controls the model shape away from the observations in the various approaches. The goal of this paper is to provide some benchmark data forming a first basis for such practical comparisons on some relatively challenging geological data sets. 

% In most structural modeling software implementing the above methods, parameter default values are generally chosen using heuristics. One of the goal of this benchmark is to assess the method's behavior on a set of common data set using these default parameter values to gain insights in these particular settings. Future applications may also include parameter tuning. However, most codes allow practitioners to change these parameters by trial-and-error to adapt the modeling results based on data quality and on the geological context. This confers a strong importance to subjective expert judgment of the quality of the solution.  %As none of the above methods directly models the geological processes which generated the structures of interest, we can paraphrase George Box in acknowledging all these models are wrong, but that they are useful in practice. Their existence and their diversity reflects the various types reasoning used to approach, understand and quantify geology.

\subsection{Spatial discretization}

DcK and RBF are meshless approaches, in the sense that the solution has a high degree of smoothness and can be evaluated everywhere in space without the need for filling the domain with a mesh. Once the system has been solved, the evaluation for a given position is given directly by Eq. (\ref{eq:basis}). It only involves to compute the distance between the evaluation point and the data points. For visualization and further processing, the solution is generally evaluated at the vertices of a Cartesian grid or unstructured mesh. the cost of increasing the spatial resolution of that visualization mesh only increases linearly with the number of evaluation nodes. However, adding new data calls for globally updating the interpolation if global basis functions are present. The cost of such a computation can be alleviated by the use of efficient solvers such as the fast multipole method \citep{greengard_fast_1987}. 

In mesh-based methods, the computation is made on the discrete support at the same resolution as the visualization. New stratigraphic data can integrated efficiently by recomputing the solution locally, as the basis functions have a compact support. Visually, coarse meshes may display some discontinuities of the first derivative of the scalar field. Globally or locally adapting the mesh resolution to reduce these artifacts calls for solving the interpolation problem again at least in the refined part of the mesh \citep{Frank2007CG}. 

The above remarks may suggest that the meshless methods are always preferable to mesh-based methods, as meshless methods only depend on the data and not on a particular choice of mesh. However, as in surface-based structural modeling \citep{Caumon2009MG}, this allows modelers to adapt the level of detail based on interpretive knowledge. Also, the ultimate goal of a geological structural model is generally to simulate fluid flow, deformation or other physical phenomena in the subsurface, which sooner or later, calls for mesh generation and processing abilities in subsurface modeling systems. This even holds for model visualization, as the visualization of implicit geological surfaces is generally achieved on discrete meshes by polygonal isosurface extraction algorithms such as the marching cubes \citep{Calcagno2008PEPI} or marching terahedra method \citep{Frank2007CG}. 

\todo[inline]{GC: Did anyone try ray tracing with meshless approaches?}
\todo[inline]{GC: Reading this section again, I am not sure that all co-authors will agree on what I wrote ;-) Suggestions are welcome (as everywhere else).}

\subsection{Summary}

In the above review, we have highlighted some common principles and differences between the existing implicit interpolation methods. The purpose here is not to enter an endless philosophical debate on the appropriateness of the various methods: they all have shown their merits to create reasonable geological models from incomplete observations. Overall, the diversity of these approaches illustrates the interest of using mathematical abstractions and computational methods to help geological interpetation and reasoning \citep{frodeman_geological_1995}. %compensate for the impossibility to quantitatively integrate all available geological concepts with observations. 

The above review allows us to provide the main parameters of the methods used in this benchmark study. Indeed, we applied several implementations of DcK, RBF and DSI to the data sets described in Section \ref{sec:data}. The names, main features and references 
for the corresponding codes are summarized in Table \ref{tab:codes}. 

\begin{table}
\centerline{\rotatebox{90}{
\begin{tabular}{|p{15mm}|p{25mm}|p{15mm}|p{15mm}|p{30mm}|p{20mm}|p{17mm}|p{3.5cm}|}
\hline
\textbf{Name} & \textbf{Method} & \textbf{Compu\-ta\-tion} & \textbf{Visualiz\-ation} & \textbf{References} & \textbf{Language / dependency} & \textbf{License} & \textbf{Main parameters} \\
\hline
GemPy & DcK (increments) & Meshless & Marching cubes & \citet{delaVarga2018GMDD}, \url{https://www.gempy.org} & Python / Theano & LGPL & Type and parameters of variogram model \\
\hline
LoopStru\-ctural & DSI (piecewise linear constant gradient, finite differences) & Cartesian and tetra\-hedral & Marching cubes & In prep., \url{https://loop3d.github.io/LoopStructural} & Python & MIT & Least squares weights for data / regularization \\
\hline 
MSS & FEM / DSI & Points & Marching cubes & \citep{Renaudeau2019BEMRMX,Renaudeau2019MG} & C++ & Proprietary & Least squares weights for data / regularization.\\
\hline 
SIGMA & DSI, finite differences & Cartesian & Marching cubes & \citet{Irakarama2018EAGE}, \url{https://www.ring-team.org/software/ring-libraries/258-sigma} & C++/CUDA & RING & Least squares weights for data / regularization \\
\hline
Structural\-Lab & DSI, piecewise linear minimal gradient & Tetra\-hedral & Marching tetra & \citet{Frank2007CG,Caumon2013GaRSITo}, \url{https://www.ring-team.org/software/skua-gocad-plugins/42-structurallab} & C++ / SKUA-GOCAD & RING & Least squares weights for data / regularization \\
\hline
SurfE & DcK & Cartesian & Marching cubes & \citet{Hillier2014MG}, \url{https://github.com/MichaelHillier/surfe} & C++ & MIT & Type and coefficients of RBF functions; interpolation or approximation \\ 
\hline
\end{tabular}
}}
\caption{Main features and parameters of the computer codes used in this benchmark study.}
\label{tab:codes}
\end{table}

\todo[inline]{Table to be checked and corrected where needed}

\section{Benchmark data sets}\label{sec:data}

We have selected three geological data sets to illustrate some features that structural modeling methods typically struggle with. In particular, we considered relatively complex geometric shapes generated by different types of geological processes: folding (Hecho data, Section \ref{sec:Hecho}) and hydrothermal alteration (Moureze synthetic data, Section \ref{sec:Moureze}), which can generate convoluted surfaces and stratigraphic series. The goal of the third data set (Claudius, Section \ref{sec:Claudius}) is to test ability of modeling methods to handle lateral thickness variations. Indeed, although isopach hypothesis is often a convenient modeling assumption, stratigraphic thickness may vary quite rapidly lateraly due to the independent or combined effect of depositional and tectonic processes (e.g., growth strata, differential compaction, sedimentary clinoforms, etc.). 

For all three data sets, we propose a brief description of the geological settings, and we describe the available data. Each data set was prepared so as to serve both as a test of two-dimensional and three-dimensional codes. Therefore, each data set generally comes both with cross-section lines and unstructured point clouds.

% QUESTION: Shall we also include some (correlated or uncorrelated) noise / outliers to check the ability of methods to resist these problems? I don't think this is necessary (noise could be added in future papers, but I prefer asking). 

\subsection{Hecho: Folded lobes, Hecho Group, Aragon, Spain}
\label{sec:Hecho}

\begin{figure}
\centering\begin{tabular}{cc}
a & \includegraphics[width=0.8\textwidth]{Hecho1}\\
b & \includegraphics[width=0.8\textwidth]{Hecho2} \\
c & \includegraphics[width=0.8\textwidth]{Hecho3} \\
\end{tabular}
\caption{Folded turbidite lobes outcrop (a) and interpretation with the ``Full'' data set (b) and the ``sparse'' data set (c). 
Tangent and orientation data are shown as black lines and arrows.}
\label{fig:FLH2D}
\end{figure}

The Hecho [het'cho] data set (Fig. \ref{fig:FLH2D}) corresponds to a folded and faulted turbidite outcrop of approximately $16 \times 4$ m located close to Arag\"u\'es del Puerto, Arag\'on, Spain. The layers correspond to Eocene turbidite lobes deposited in the Pyrenean foreland basin (Hecho Group) and then affected by north/south shortening \citep{Bastida2012JSG}. The interbedding of sandstone and shale beds is a source of mechanical heterogeneities which affected the localization of deformation during folding. On this outcrop, most of the shortening is accommodated by flexural slip of sandstone beds; shale beds follow a more ductile deformation between sandstone beds. The fold style (vergence, aperture) changes laterally and vertically, and probably results from variations in the relative thicknesses for the two main rock types. Some minor faults are visible and interpreted on the outcrop. The faults are provided as polygonal lines and triangulated surfaces extrapolated orthogonally 50 cm to the outcrop picture. The mesh resolution of the fault surfaces is approximately 15 cm. The sand lobes only display minor thickness variations at the scale of the outcrop, but some mild apparent thickness variation appears due to perspective. We chose to keep this artifact in the benchmark data set, as thickness variations management challenges have been documented for some implicit structural modeling methods \citep{Laurent2016EaPSL}. 

Nine horizons (H1 to H9) were interpreted on the image plane (XZ) as polygonal lines (Fig. \ref{fig:FLH2D}b). For each of these lines, a random Y coordinate was sampled from a uniform law between -0.5 and 0.5 m to avoid singularities in the three-dimensional case. Some stratal traces and the associated slopes were also obtained by picking some segments parallel to the bedding on the surface. In addition to the ``full'' interpretation, a sparse interpretation is provided to test the ability of the methods to extrapolate away from the data (Fig. \ref{fig:FLH2D}c). This sparse data set only contains the faults and the horizons H1, H2, H4, H5, H7 and H9; only the subhorizontal part of H4 and H5 is provided. As a result, a significant data gap exists at the center of the outcrop between H7 and H2, where the observed layers are overall subvertical.

As some of the implicit modeling methods start with imposed values at the scalar field, we have proposed reference scalar field values (relative geological time) to be equal to a rough estimate of the layer thickness on the interpretation points (Table\ref{tab:Hecho}).

\begin{table}
\centerline{\begin{tabular}{|l|c|c|c|c|c|c|c|c|c|}
\hline
Horizon                       & H1 & H2   & H3 & H4  & H5  & H6 & H7   & H8  & H9  \\
\hline
Scalar value (arbitrary unit) & 0. & 0.78 & 1.15 & 1.9 & 2.5 & 3.1 & 3.9 & 4.4 & 5.2 \\
\hline
\end{tabular}}
\caption{Approximate thicknesses and relative geological time values values chosen for the Hecho Data Set. }
\label{tab:Hecho}
\end{table}

The main goal of this data set is to test the ability of the structural modeling methods to honor conformable series affected by strong and laterally variable folds. 

\subsection{Claudius Data: carbonate mounds from NE Australia}
\label{sec:Claudius}

\begin{figure}
\centering\begin{tabular}{cc}
a & \includegraphics[width=0.7\textwidth,height=6cm]{Claudius}\\
b & \includegraphics[width=0.8\textwidth]{Claudius1} \\
\end{tabular}
\caption{Interpreted stratigraphic series in the Claudius data set: sections (a) and points in side view (b). A is in blue, B in green, C in yellow and D in violet. In b, the Z Color scale bounds are set for each horizon. Orientation data (normal vectors) in black.}
\label{fig:ClaudiusData}
\end{figure}
\todo[inline]{Add legend and a figure with the reference surfaces and the data only + size details}

The second data set was taken from the offshore Carnarvon Basin, NW Australia. The Claudius area is imaged with a 3D seismic survey acquired by WesternGeco and provided by Geoscience Australia. We extracted a small stratigraphic section that displays several carbonate mounds and the associated slope and offshore deposits (Fig. \ref{fig:ClaudiusData}). We interpreted four main horizons named A to D. The most shallow horizon A is relatively flat, whereas D (the top of upper triassic carbonate build-ups) displays significant relief. The intermediate horizons B and C are slightly deformed maybe due to differential compaction. Overall, significant layer thickness variations exist between these stratigraphic surfaces. The detail of the seismic data set shows multiple slope deposits and onlaps, but at the scale of interest, we choose to consider here that the series is globally conformable. The isochore (vertical) thickness of the layer bounded by the C and D seismic reflectors varies between 131~m and 966~m, with an average of 598~m. 

As in the Hecho data set, we did a rough estimate of average layer thicknesses to choose the reference relative geological time values given in Table \ref{tab:Claudius}. Values approximately correspond to the two-way time thickness (in ms) in seismic data. The vertical depth coordinate (in meters) was obtained by multiplying by three the two-way seismic travel time (in millisecond). This value corresponds to a very slow medium and is clearly inaccurate, but it allows to easily depth convert the seismic image for testing the results, and to increase the relative thickness variations between the various reflectors.  

\begin{table}
\centerline{\begin{tabular}{|l|c|c|c|c|c|c|c|c|c|}
\hline
Horizon                       & A   & B   & C   & D    \\
\hline
Approx. thickness to A (m)    & 0   & 180 & 750 & 1000 \\
\hline
Scalar value (arbitrary unit) & 0 & 60 & 250 & 320      \\
\hline
\end{tabular}}
\caption{Approximate thicknesses and relative geological time values values chosen for the Claudius Data Set. }
\label{tab:Claudius}
\end{table}

The main goal of this data set is to test the ability of implicit methods to model a stratigraphic series having strong thickness variations with one single continuous scalar field. 

\subsection{The Moureze 3D synthetic surface}
\label{sec:Moureze}


\begin{figure}
\centering\includegraphics[width=\textwidth]{Moureze}
\caption{Synthetic complex surface generated by a perturbed distance field.}
\label{fig:Moureze}
\end{figure}
\todo[inline]{Add legend and a figure with the reference surfaces and the data only + size details}

To study the behavior of implicit interpolation methods for very complex structural geometries as encountered in salt tectonic studies and ore body delineation, we created a synthetic reference surface using the ODSIM (Object-Distance Simulation) method \citep{Henrion2010MG}. This method was initially introduced to model karsts and hydrothermal alteration bodies \citep{Henrion2008PEIGC,Rongier2014G} and was recently adapted to generate salt surfaces \citep{Clausolles20188ECE2}. In this study, we emulated a hydrothermal alteration body by defining a basement where several subvertical fractures are rooted. The fractures and the basement were defined by manual picking. The distance to the basement and the fracture set was then computed and perturbed by a spatially correlated random field (normal distribution of average 10 m and standard deviation equal to 36 m and Gaussian variogram model of a 36 m horizontal range (isotropic) and 20 m vertical range). Finally, the surface was obtained 
by thresholding (value of 28 m) the perturbed distance field and extracting the surface using Marching Cubes. Some additional cleaning was made to remove isolated bubbles and locally smooth and locally perturb (manually) the obtained geometry. This reference surface is made of one single connected component, but it has three handles. 

From this surface, we extracted two sets of seven E/W and thirteen N/S parallel cross-sections. These sections only contain polygonal lines, some of which consist of several connected components. We also created an unstructured point set by decimating the nodes of the reference surface. The decimation strategy was performed both by random sampling and by manual removal of nodes. The manual removal targeted some bumps so that only a very limited number of points remained to constrain the surface geometry. The resulting data points have a heterogeneous density in order to evaluate at once the ability of the reconstruction methods to retrieve the reference shape, both in the case of dense and sparse sampling conditions. A (randomly chosen) fraction of these nodes also provides information about the reference surface orientation. Both the section lines and the points are exactly on the reference surface, so data noise can be considered as negligible. 


\section{Results and evaluation Metrics}
\label{sec:results}

\subsection{Results}
The interpolation results for the various data sets and codes are available on the benchmark repository accessible at \url{https://github.com/Loop3D/ImplicitBenchmark}. Each data set is stored in a separate directory. Sub-directories contain results obtained by each method, in the form of the scalar field, together with the information to reproduce these results (Jupyter notebook, software version, script and/or parameter set, ReadMe file). The same reference mesh / evaluation resolution was used for all benchmark results to facilitate comparisons. 

For each data set, the directory \texttt{AllResults} contains summary tables corresponding to some statistics computed for all results. (As of July 31st, 2020, the tables have only been computed for the Claudius data set). 

\subsection{Metrics}
To evaluate the results, we have considered three aspects: (1) the degree of fit to the input data; (2) the predictivity, considering the fitness to data not included in the benchmark; and (3) intrinsic measures of the produced scalar fields based on topological considerations. We now describe and briefly discuss these metrics. 

\subsection{Degree of data fit and predictivity}

All interpolation methods aim at honoring the input data, but we have seen that many methods also approximate the input data. Such approximations allow for filtering geometric noise and are also very useful to avoid spurious extrapolation artifacts. 
Measuring the degree of extrapolation is, therefore, not necessarily bringing much insight in general, as the nugget effect on DcK or the weight of the regularization terms in least-squares or physically based approaches can be tuned to decrease the error. However, this may lead to overfitting and to a lower predictive power of the models away from the data. Therefore, we consider, for each data sets, some test data which can be used to test the ability of the models to predict the structural geometry away from the interpreted points. For this, we consider several types of test data:

\begin{enumerate}
	\item Test points located on the input interfaces allow to test the lateral predictive ability of the various methods.  
  \item Test points located on interfaces not considered in the interpolation data allow to test the ability of implicit modeling to exploit stratigraphic consistency to extrapolate some horizons from others. 
  \item Test orientations, located on data points or not, allow to test the ability of the methods to forecast stratal orientation. 
\end{enumerate}

In the case of the sythetic Moureze data set, these test points can be easily selected from the reference surface. For the Hecho and Claudius data, we may similarly use some of the reference interpretations surfaces, for instance the normal or som sample points not provided as part of the data set. This, however, remains subject to the accuracy of the initial interpretation in the first place. Additional interfaces can also be interpreted from the outcrop picture and the reflection seismic data, with again the same restrictions that these interpretations are themselves an approximation of the true earth structures. They are, nonetheless, open to scrutiny and should in principle be a more accurate model than interpolated solutions.

\subsubsection{Orientation errors}

The orientation error $E_o$ is evaluated at each validation or test point $\bx_i$ from the scalar product between the reference orientation (represented by the normal unit vector $\bn_{ref}$) and the normalized gradient of the scalar field:

\begin{equation}
\label{eq:erro}
E_o(\bx_i) = \cos^{-1} \left( \bn_{ref} \cdot \frac{\nabla f(\bx_i)}{||\nabla f(\bx_i)||} \right).
\end{equation}

In this case, the normalization of the scalar field gradient produces a homogeneous error measure for all methods, up to the accuracy of the gradient approximation on the evaluation grid. For this, we used a first order centered finite difference scheme for the methods which evaluate the scalar field on a Cartesian grid. In this case, care must be taken to ignore boundary voxels where such the gradient estimate is inaccurate. For StructuralLab, which computes and evaluates the scalar field on a piecewise tetrahedral mesh, the gradient is piecewise constant as it is estimated using the same basis functions. 

The orientation error gives a direct estimation of the local solution quality, but is does not measure all the aspects of its global accuracy. For instance, adding a small low-frequency trend to the scalar field would not significantly change the orientation error, but could more significantly affect the geometry of the isosurfaces. Therefore, computing the deviation in term of scalar field values can also be useful. 

\subsubsection{Scalar value difference and distance to the level set}

For the data and test points, computing the root mean squared error between the reference and the interpolated scalar field is possible for all methods. However, for methods that impose the values (Eq. (\ref{eq:datasystem})), the result depends on the initial (and somewhat arbitrary) choice of the initial scalar field values. In the increment-based methods, the values can be very different depending on the parameters of the covariance / basis function, making comparisons of absolute scalar field values meaningless. Therefore, we have tried to normalize the scalar field by dividing the value difference by the inverse of the local scalar field gradient norm $||\nabla f||$ to approximate a metric error \citep{Caumon2010MG}: 

\begin{equation}
\label{eq:errf}
E_f(\bx_i) = \frac{f_{ref} - f(\bx_i)}{||\nabla f(\bx_i)||},
\end{equation}

\noindent where $\bx_i$ is the test data point location, $f_{ref}$ the reference scalar field value at that location. When the reference value is not known (with increment methods and with test data located on new horizons), we estimate it as the average of the scalar field on for the current horizon data. 

When the gradient norm is locally constant, Eq (\ref{eq:errf}) gives the true distance between the data point and the interpolated level set $f(\bx) = f_{ref}$. However, thickness variations exist in some data sets, so the error $E_f$ only approximates the distance in that case, and the accuracy decreases when the true distance increases. Also, as no reference scalar field exists, Eq. \ref{eq:errf} necessarily uses the norm of the interpolated scalar field to assess the error, so the degree of distance approximation may change from one solution to the next. Finally, the gradient needs to be computed on the evaluation grid, which is a Cartesian mesh or a tetrahedral mesh, which introduces additional approximations. Indeed, the gradient is piecewise constant for tetrahedral piecewise linear codes, whereas it is computed by finite difference approximation on Cartesian evaluation grids; using a coarse grid resolution can, therefore, increase the errors for meshless implicit methods as compared to the gradient they actually produce.  

For these reasons, we also extract the level set corresponding to reference scalar value $f_{ref}$ and explicitly compute the error $E_d$ corresponding to the signed distance between the data point and that level set: 

\begin{equation}
\label{eq:errd}
E_d(\bx_i) = \underset{\bx}{\mathrm{argmin}} \left(\bx_i - \bx \right), \mbox{ with } \bx | f(\bx) = f_{ref}.
\end{equation}

This computation accurately estimates the true orthogonal distance to the reference level set, except in some boundary cases where the closest isosurface point lies on the boundary of the modeling domain.  

Out of curiosity, we have assessed experimentally the difference between $E_d$ and $E_f$ on the Claudius model. For example, the correlation coefficients for GemPy between both approaches vary between 0.999 on Horizon A, down to 0.7 on Horizon D, where and some very large relative differences are observed (locally  more than 100\% error for the gradient-based distance). This confirms that the distances $R_d$, even though they are more costly to evaluate, should be preferred to the $E_f$ approximation in general.  

On the data repository, statistics of both signed distance errors are provided for the benchmark data and for some additional test data. The SKUA macros to compute these errors on the various results are also given. Bias in the solution can be evaluated by centrality statistics (the mean or median signed distance should be zero), whereas the accuracy can be evaluated using the amplitude of dispersion statistics (standard deviation, interquartile range, etc.). 

However, rather than gobal statistics, we plan to also use these error measures to map the differences between the various solutions. Although the number of models is not large enough, this could provide an interesting view about where data lead to very similar solutions for the various methods, and where the differences increase to reach maximal uncertainty.  


\subsection{Topology of the reconstructed series}

A limitation of the above distance-based errors is that they only consider a set of points and particular geological interfaces. Therefore, statistics obtained with these distances only provide a limited view on the produced relative geological time values. 

For stratigraphic series, a simple idea to globally check the elementary correctness of the results is to make sure that the interpolated scalar fields have no internal local extrema. Indeed, an internal local extremum implies closed isosurfaces around that extremum, which cannot be produced by stratigraphic processes. As a result, local extrema should only exist on the boundary of the domain of study in stratigraphic settings. Note, however that closed surfaces or surfaces with handles may exist in some cases which involve complex deformation (e.g., boudinage, salt tectonics or other ductile deformation patterns), objects generated by hydrochemical processes (e.g., hydrothemal bodies, or karsts) or at scales generally not considered in geomodeling (e.g., oolithes, fossils). 

Characterizing such characteristics directly on scalar fields can be made by computing the Betti numbers $\beta_0, \beta_1, \beta_2$ of the isosurfaces: $\beta_0$ is the number of connected components, $\beta_1$ is the number of surface handles, and $\beta_2$ the number of internal holes. Rather than extracting all possible level sets to compute these characteristics, one can use concepts from persistent homology and topological data analysis for this \citep[e.g.,][]{Chazal2017ACMS,Tierny2017,Wasserman2018}. The main idea of persistent homology is to evaluate when the topology of groups of isosurfaces change. For this, one can start by computing local extrema and some field lines (parallel to the scalar field gradient) connecting theses local extrema. Field files may meet at saddle points, which are the equivalent of a pass on a topographic surface (a pass is characterized by a negative Gaussian curvature --positive maximum curvature and negative minimum curvature). Therefore, the set of critical points (local minima, extrema and saddle points) can be used to compactly characterize some essential features of the topology of the interpolated scalar fields. The critical point of the elevation of an implicit surface also provide a measure of its topological and geometrical complexity. 

For this study, we implemented the critical points on a piecewise linear tetrahedral meshes, following some ideas summarized in \citet{Tierny2017}. As a consequence, all scalar fields were transferred on the nodes of the reference tetrahedral mesh (by bilinear interpolation) before extracting critical points. This resampling may induce a small bias due to possible smearing effects. As a consequence, the estimated number of extrema available in the repository may be slightly under-estimated. 
This could be addressed in the future by using higher resolution evaluation meshes for the various solutions. 

This problem, together with an evaluation of the importance of each local extremum, could also be handled by using the filtering concept in persistent homology. Indeed, filtering allows to see how critical points vanish when the noise level in the scalar field decreases. In other terms, some local extrema correspond to some oscillations of very small amplitude, hence can removed from the analysis. We consider using this principle (and other topological data analysis concepts) in the future to extend the set of metrics used to characterize the obtained implicit models. 

In the repository, we have computed the local extrema and saddle points (critical points) for the various solutions. In the Claudius data set, we observe that all the methods yield local extrema inside the domain, which highlights their inconsistency with elementary stratigraphic principles. This comes from the thickness variation present in the data, which generates instabilities in the interpolated solution. When the data are not affected by noise, the classical approaches to address this problem is to use several scalar fields, to split the solution domain into subdomains where data and modeling hypotheses are more compatible. This approach is illustrated for instance in the Claudius data set GemPy Jupyter notebook, which obtains a consistent solution using two scalar fields. 


%\subsection{Performance}
 
\todo[inline]{Explain why we dropped performance from the discussions. Everyone ran tests on a different machine, and codes have very different features (parallel or not, 2D or 3D). Scalability can be compared nonetheless by fitting a law regression between the number of points and the computational time. THis would call for everyone to apply the interpolation with decimated data sets (e.g. using 1/100, 1/50, 1/10, 1/5, 1/2 of the points)... The time ratio between model preparation (e.g., meshing), interpolation (solving the system) and visualization (marching cubes or the like) should also be considered. Indicate that indicative run times are included in the repo.  }


\section*{Conclusions and discussions}
\label{sec:conclu}
We have made a review of implicit structural modeling methods and described the main features and parameters for a set of existing implementations. In most practical studies, these methods use several scalar fields to model a particular geological domain, so as to account for faults, stratigraphic unconformities and other types of geological dicontinuities \citep{Wellmann2018AiG}. However, to gain insights about the various approaches, we have proposed three data set where only one scalar function can be used to approximate a geological boundary (Moureze) or a series of stratigraphic surfaces (Hecho and Claudius) from relatively sparse data. 
These data sets were designed to test both two-dimensional and three-dimensional implicit modeling codes. They are heterogeneous in size and in types of geological features, showing a highly conformable but strongly folded stratigraphic series (Hecho), sedimentologically-controlled mini-basins involving fast changes of lateral stratigraphic thickness (Claudius) , and a complex fracture-controlled hydrothermal surface (Moureze). 

We have independently applied six implementations of the implicit modeling methods to these three data sets and uploaded them to a public repository. All authors used default parameters and a common grid resolution for visualization (and, when relevant computation). Finally, we proposed some criteria to evaluate the benchmark results. 

Overall, the obtained results provide an interesting way to address structural uncertainty evaluation on the given data, as the methods, applied independently by the co-authors, provide solutions which are similar in some areas and dissimilar in others. This means that the isosurfaces obtained with the various solutions could be deemed to actually represent a more objective space of solutions than that obtained with just one particular method. More generally, the use scalar field values opens alternative ways to discrete entropy to visualize stratigraphic uncertainty independently of the chosen number of layers. Persistent homology considered here as an analysis metric, also seems like a very interesting way forward to summarize some features of implicit structural models.  

The results are still under scrutiny, so this paper will be extended to include new findings of the analysis. The main learning at this stage is that none of the modeling methods used in this study is able to get a consistent stratigraphic scalar field solution for the Claudius data set. This confirms the need to continue doing research on ways to better handle thickness variations in future approaches. 

Again, we strongly discourage the use of proposed metrics for a general ranking of methods. Indeed, default parameters have been used and some arbitrary choices affecting the results have been made. Moreover, the data sets and the test data set are affected by interpretation errors; unconscious bias from the first author, who has more experience in some codes and methods than others, may also have be present when he prepared the data. Finally, the evaluation metrics are incomplete and neglect very important aspects in practice such as user interface and computational performance. The also concern one single scalar field, and do not consider fault management or how several scalar fields are combined to model complex geological domains. At a philosophical level, it is, in any case dubious to compare methods based on applications. A mathematical method may be judged based on the principles and concepts it uses. Results can only help to analyze how these principles and concepts interact with data and parameters to produce some solution. Therefore, the only very clear conclusion that we can make at this point is that further research is still needed to parsimoniously integrate geological knowledge in implicit structural modeling. 

\section*{Acknowledgments}

This work was performed in the frame of the LOOP project (\url{https://loop3d.org/}) and the RING project (http://ring.georessources.univ-lorraine.fr/). We would like to thank for their support the industrial and academic sponsors of the RING-GOCAD Consortium managed by ASGA, and the LOOP project partners. We would also like to thank Geoscience Australia for providing the Claudius seismic data set and Emerson for the SKUA-GOCAD software used to prepare the data and analyze the results. 

% Set your bibliography file here (without .bib)
\bibliography{biblio}

\end{document}

%------------------------------------------
